\documentclass{article}  % Tipo di documento
\usepackage[utf8]{inputenc} % Per caratteri accentati
\usepackage[italian]{babel} % Lingua italiana
\usepackage{amsthm}
\usepackage{amssymb}
\usepackage{amsmath}  % simboli matematici avanzati
\usepackage{xcolor} % Per i colori
\usepackage{titlesec} % Per personalizzare i titoli
\usepackage{tikz}
\usetikzlibrary{mindmap,trees}
\usepackage[most]{tcolorbox}
\usepackage{subcaption}  % per avere subfigure
\tcbuselibrary{theorems}
\usepackage{tikz}
\usetikzlibrary{automata, positioning, arrows}
\tcbuselibrary{breakable}
\usepackage{graphicx}
\usepackage[table]{xcolor} % da mettere nel preambolo
\usepackage{mathrsfs} % https://www.ctan.org/pkg/mathrsfs
\graphicspath{ {./media/} }
\usepackage{centernot}
\usepackage[hidelinks]{hyperref}
\usepackage{cancel}


\newtcbtheorem[no counter]{theorem}{Teorema}%
{colback=blue!5, 
colframe=blue!50!black, 
fonttitle=\bfseries,
    breakable,            % permette di spezzare il box su più pagine
    enhanced,
    break at=0pt}{}

\newtcbtheorem[no counter]{definition}{Definizione}%
{colback=magenta!5, 
colframe=magenta, 
fonttitle=\bfseries,
    breakable,            % permette di spezzare il box su più pagine
    enhanced,
    break at=0pt}{}

% Imposto colore delle subsection
\titleformat{\subsection}
  {\normalfont\large\color{red}} % stile del titolo
  {\thesubsection}{1em}{} % numerazione e spaziatura

% Definiamo un nuovo ambiente per gli esempi
\newtcolorbox{esempio}[1][]{
    colback=white,       % colore di sfondo
    colframe=gray,       % colore del bordo
    fonttitle=\bfseries,
    title=#1,
    boxrule=0.5pt,       % spessore del bordo
    arc=4pt,             % angoli arrotondati
    left=4pt, right=4pt, top=4pt, bottom=4pt,
	breakable,            % permette di spezzare il box su più pagine
    enhanced,
    break at=0pt
}

\newtcolorbox{esercizio}[1][]{
    colback=white,       % colore di sfondo
    colframe=green!60!black,       % colore del bordo
    fonttitle=\bfseries,
    title=#1,
    boxrule=0.5pt,       % spessore del bordo
    arc=4pt,             % angoli arrotondati
    left=4pt, right=4pt, top=4pt, bottom=4pt,
    breakable,            % permette di spezzare il box su più pagine
    enhanced,
    break at=0pt
}

\newtcolorbox{osservazioni}[1][]{
    colback=white,       % colore di sfondo
    colframe=yellow!80!orange,       % colore del bordo
    fonttitle=\bfseries,
    title=#1,
    boxrule=0.5pt,       % spessore del bordo
    arc=4pt,             % angoli arrotondati
    left=4pt, right=4pt, top=4pt, bottom=4pt,
    breakable,            % permette di spezzare il box su più pagine
    enhanced,
    break at=0pt
}

% Creo un nuovo ambiente "ragionamento" senza quadratino
\newenvironment{ragionamento}[1][]
  {\begin{proof}[Ragionamento#1]\renewcommand{\qedsymbol}{}\normalfont}
  {\end{proof}}

\title{Analisi 2}
\author{Ede Boanini}
\date{\today}

\begin{document}
\maketitle
\tableofcontents % genera automaticamente l’indice
\newpage
\footnotesize
%%%%%%%%%%%%%%%%%%%%%%%%%%%%%%%%%%%%%%%%%%%%%%%%%%%%%%%%%%%%%%%%%%%%%%%%%%%%%%%%%%%%%%%%%%%%%%%%%%%%%%
\section{Equazioni differenziali}
\subsection{EDO di I ordine}
\[
	y'=3e^{x}
\]
\subsection{EDO di II ordine}
\[
	2y''+5y=e^x
\]
\subsubsection{EDO di II ordine omogenee}
\subsubsection{EDO di II ordine non omogenee}
\subsubsection{Metodo di somiglianza per EDO II ordine non omogenee}
Utile per trovare la soluzione particolare $y_p$ delle EDO di II ordine non omogenee.
\[
	ay''+by'+cy=f(x)
\]
\textcolor{red}{$f(x)$ è un polinomio di grado $n$}:
\begin{esercizio}[Caso 1: $c \neq 0$]
	\textcolor{red}{$c \neq 0$}
	\[
		y''+2y'-8y=x^2+3x+1
	\]
	Qui $f(x)=x^2+3x+1$ che è un polinomio con
	\begin{itemize}
		\item $a=1$
		\item $b=2$
		\item $c=-8$
	\end{itemize}
	$c \neq 0$ quindi devo cercare polinomio di grado $n$ (che corrisponde al grado del polinomio a destra dell'uguale), quindi $n=2$:
	\begin{enumerate}
		\item Scrivo polinomio di grado $n=2$ generico:
		      \[
			      \alpha x^2+\beta x + \gamma
		      \]
		      Questa sarà la forma della soluzione particolare $y_p$
		\item Devo trovare chi sono $\alpha,\beta,\gamma$, come faccio?
		      Calcolo derivata prima e seconda della soluzione particolare: $y_p=\alpha x^2+\beta x + \gamma$ \\
		      \begin{itemize}
			      \item Calcolo derivata prima:
			            \[
				            y'_p=2 \alpha x+\beta
			            \]
			      \item Calcolo derivata seconda:
			            \[
				            y''_p=2 \alpha
			            \]
		      \end{itemize}
		\item Sostituisco $y_p,y'_p, y''_p$ trovati in $y''+2y'-8y$: \\
		      Ho:
		      \[
			      y''+2y'-8y=x^2+3x+1
		      \]
		      Sostituisco $y_p=y,y'_p=y', y''_p=y''$
		      \[
			      (2 \alpha)+2(2 \alpha x+\beta)-8(\alpha x^2+\beta x + \gamma)=x^2+3x+1
		      \]
		      Ora sviluppo i calcoli per trovare $\alpha,\beta,\gamma$:
		      \begin{align*}
			      (2 \alpha)+2(2 \alpha x+\beta)-8(\alpha x^2+\beta x + \gamma)      & = x^2+3x+1 \\
			      2 \alpha+4 \alpha x+2\beta-8 \alpha x^2 -8 \beta x -8 \gamma       & = x^2+3x+1 \\
			      x^2(-8 \alpha)+x(4 \alpha -8 \beta) + 2 \alpha + 2\beta - 8 \gamma & = x^2+3x+1
		      \end{align*}
		      Metto a sistema: \\
		      \textcolor{red}{$x^2(-8 \alpha)$}$+$\textcolor{blue}{$x(4 \alpha -8 \beta)$} $+$ \textcolor{orange}{$2 \alpha + 2\beta - 8 \gamma$} $=$ \textcolor{red}{$x^2$} $+$ \textcolor{blue}{$3x$} $+$ \textcolor{orange}{$1$}
		      \[
			      \begin{cases}
				      -8 \alpha = 1          \\
				      4 \alpha - 8 \beta = 3 \\
				      2 \alpha + 2 \beta - 8 \gamma = 1
			      \end{cases}
		      \]
		      Risolvo il sistema e ottengo:
		      \[
			      \begin{cases}
				      \alpha = - \frac{1}{8} \\
				      \beta = - \frac{7}{16} \\
				      \gamma = - \frac{17}{64}
			      \end{cases}
		      \]
		\item La soluzione particolare $y_p=\alpha x^2+\beta x + \gamma$ dopo aver trovato $\alpha, \beta, \gamma$ sarà:
		      \begin{align*}
			      y_p
			       & = \alpha x^2+\beta x + \gamma                        \\
			       & = - \frac{1}{8} x^2 - \frac{7}{16} x - \frac{17}{64}
		      \end{align*}
	\end{enumerate}
\end{esercizio}

\begin{esercizio}[Caso 2: $c$ uguale a zero]
	\textcolor{red}{$c = 0, \quad b \neq 0$}
	\[
		y''+7y'=x+3
	\]
	$c=0$ quindi devo cercare polinomio di grado $n+1$, poichè il grado di $x+3$ è 1, allora $n+1=1+1=2$, quindi $n=2$:
	\begin{enumerate}
		\item Scrivo polinomio di grado $n=2$ generico:
		      \[
			      y_p=\alpha x^2+\beta x + \gamma
		      \]
		\item Calcolo derivata prima e seconda
		      \begin{itemize}
			      \item $y'_p=2\alpha x+\beta$
			      \item $y''_p=2\alpha$
		      \end{itemize}
		\item Sostituisco nella EDO originale e trovo $\alpha, \beta, \gamma$: \\
		      Sostituisco $y_p,y'_p, y''_p$ trovati in $y''+7y'$:
		      \[
			      y''+7y'=x+3
		      \]
		      \[
			      (2 \alpha)+7(2\alpha x+\beta)=x+3
		      \]
		      \[
			      2 \alpha+14\alpha x+7\beta=x+3
		      \]
		      \[
			      \textcolor{red}{x(14\alpha)}+\textcolor{blue}{2\alpha+7\beta}=\textcolor{red}{x}+\textcolor{blue}{3}
		      \]
		      Metto a sistema:
		      \[
			      \begin{cases}
				      14 \alpha = 1 \\
				      2 \alpha + 7 \beta =3
			      \end{cases}
		      \]
		      Risolvo il sistema e ottengo:
		      \[
			      \begin{cases}
				      \alpha = \frac{1}{14} \\
				      \beta = \frac{20}{49}
			      \end{cases}
		      \]
		\item Ottengo la soluzione particolare: \\
		La soluzione particolare $y_p=\alpha x^2+\beta x + \gamma$ dopo aver trovato $\alpha, \beta, \gamma$ sarà:
		      \begin{align*}
			      y_p
			       & = \alpha x^2+\beta x + \gamma                        \\
			       & = \frac{1}{14} x^2 + \frac{20}{49} x - 0 \\
				   & = \frac{1}{14} x^2 + \frac{20}{49} x
		      \end{align*}
	\end{enumerate}
\end{esercizio}


\textcolor{red}{$f(x)$ è un esponenziale} \\
\textcolor{red}{$f(x)$ è $\sin(x)$ o $\cos(x)$} \\
%%%%%%%%%%%%%%%%%%%%%%%%%%%%%%%%%%%%%%%%%%%%%%%%%%%%%%%%%%%%%%%%%%%%%%%%%%%%%%%%%%%%%%%%%%%%%%%%%%%%%%
\section{Calcolo infintesimale per le curve}
\subsection{Calcolo vettoriale}
\begin{definition}{Vettore}
	Il vettore $\vec{x} \in \mathbb{R}^n$ è una n-upla:
	\[
		\vec{x}=\left(x_1,x_2,\cdots,x_n\right)
	\]
\end{definition}
\subsubsection{Norma di un vettore e proprietà associate}
\begin{definition}{Norma di un vettore}
	SSia $\vec{x} \in \mathbb{R}^n$ un vettore. Indico la \textcolor{red}{norma/lunghezza} di un vettore come la radice quadrata del prodotto scalare del vettore $\vec{x}$ per se stesso
	o equivalentemente, la radice quadrata della somma dei suoi componenti al quadrato, il numero reale non negativo:
	\[
		\textcolor{red}{\|\vec{x}\|}=\sqrt{\langle \vec{x}, \vec{x} \rangle}=\sqrt{x_1^2+x_2^2+ \cdots + x_n^2}=\sqrt{\sum_{i=1}^{n}(x_i)^2}
	\]
\end{definition}
\begin{theorem}{Formula di Carnot}
	SSiano $\vec{x},\vec{y} \in \mathbb{R}^n$ due vettori. La norma al quadrato della somma dei due vettori equivale alle loro norme al quadrato più due volte il loro prodotto scalare
	\footnote{Un pò come $(a+b)^2=a^2+b^2+2ab$}:
	\[
		\textcolor{red}{\left\|\vec{x}+\vec{y}\right\|^2}=\left\|\vec{x}\right\|^2+\left\|\vec{y}\right\|^2+2\langle \vec{x}, \vec{y} \rangle
	\]
	\begin{proof}
		Sapendo che $\|\vec{x}\|=\sqrt{\langle \vec{x}, \vec{x} \rangle}$, allora:
		\begin{align*}
			\textcolor{red}{\left\|\vec{x}+\vec{y}\right\|^2}
			 & = \left(\sqrt{\langle \vec{x} + \vec{y}, \vec{x} + \vec{y} \rangle}\right)^2                                                                \\
			 & = {\langle \vec{x} + \vec{y}, \vec{x} + \vec{y} \rangle}                                                                                    \\
			 & \textcolor{cyan}{\text{per bilinearità del prodotto scalare,}}                                                                              \\
			 & \textcolor{cyan}{\text{come se facessi }(a+b)(a+b)=a^2+ab+ba+b^2}                                                                           \\
			 & = \langle \vec{x}, \vec{x} \rangle + \langle \vec{x}, \vec{y} \rangle + \langle \vec{y}, \vec{x} \rangle + \langle \vec{y}, \vec{y} \rangle \\
			 & = \|\vec{x}\|^2+ 2\langle \vec{x}, \vec{y} \rangle + \|\vec{y}\|^2                                                                          \\
			 & = \|\vec{x}\|^2 + \|\vec{y}\|^2 + 2\langle \vec{x}, \vec{y} \rangle
		\end{align*}
	\end{proof}
\end{theorem}
\begin{theorem}{Disuguaglianza di Cauchy-Schwartz p.56}
	SSia
\end{theorem}
\subsubsection{Funzione "lunghezza" di un vettore}
\begin{theorem}{Disuguaglianza triangolare p.58}
	QQuesto teorema usa chauchy Schwartz e formula di carnot per la dimostrazione
\end{theorem}
\subsection{Spazio metrico}
\begin{definition}{Distanza euclidea}
	SSiano $\vec{x},\vec{y} \in \mathbb{R}^n$ due vettori. La \textcolor{red}{distanza euclidea} tra $\vec{x}$ e $\vec{y}$ è il numero reale non negativo definito dalla norma della loro differenza:
	\[
		d(\vec{x},\vec{y})=\left\|\vec{x}-\vec{y}\right\|=\sqrt{\langle \vec{x}-\vec{y}, \vec{x}-\vec{y} \rangle}=\sqrt{\sum_{i=1}^{n}(x_i-y_i)^2}
	\]
\end{definition}
\begin{definition}{Palla aperta}
	S\begin{itemize}
		\item Sia $\vec{x_0} \in \mathbb{R}^n$ fissato
		\item Sia il raggio $r>0$ dove $r \in \mathbb{R}$
	\end{itemize}
	Si definisce \textcolor{red}{palla aperta} o intorno sferico di centro $\vec{x_0}$ e raggio $r$, come l'insieme:
	\[
		B(\vec{x_0},r)=\left\{\vec{x} \in \mathbb{R}^n \mid d(\vec{x}, \vec{x_0})<r\right\} \subseteq \mathbb{R}^n
	\]
\end{definition}
\begin{definition}{Insieme limitato p.64}
	SSia
\end{definition}
\begin{definition}{Punto interno p.64}
	SSia
\end{definition}
\begin{definition}{Punto esterno p.64}
	SSia
\end{definition}
\begin{definition}{Punti di frontiera p.64}
	SSia
\end{definition}
\begin{definition}{Punto di accumulazione p.65}
	SSia
\end{definition}
\begin{definition}{Insieme aperto p.65}
	SSia
\end{definition}
\begin{definition}{Insieme chiuso p.65}
	SSia
\end{definition}
\begin{theorem}{Continuità p.74}
	SSia
\end{theorem}
\begin{theorem}{Criterio del confronto p.77}
	SSia
\end{theorem}
\break
%%%%%%%%%%%%%%%%%%%%%%%%%%%%%%%%%%%%%%%%%%%%%%%%%%%%%%%%%%%%%%%%%%%%%%%%%%%%%%%%%%%%%%%%%%%%%%%%%%%%%%
\section{Calcolo differenziale per funzioni in più variabili}
\begin{theorem}{Curva regolare p.103}
	SSia
\end{theorem}
\begin{theorem}{Definizione di Derivabilità con il vettore gradiente p.111}
	SSia
\end{theorem}
\begin{theorem}{Derivata Direzionale p.112}
	SSia
\end{theorem}
\subsection{Differenziabilità}
\subsubsection{Derivabilità vs Differenziabilità}
\begin{theorem}{Definizione Differenziabilità p.115}
	SSia
\end{theorem}
\begin{theorem}{Definizione Differenziale p.115}
	SSia
\end{theorem}
\begin{theorem}{Definizione Differenziabilità e Continuità p.121}
	SSia
\end{theorem}
\begin{theorem}{Teorema del differenziale totale p.122}
	SSia
\end{theorem}
\begin{theorem}{Regola della catena p.124}
	SSia
\end{theorem}
\begin{theorem}{Proprietà del differenziale p.125}
	SSia
\end{theorem}
\begin{theorem}{Derivazione della funzione composta p.128}
	SSia
\end{theorem}
\begin{theorem}{Teorema di Schwartz p.132}
	SSia
\end{theorem}
\begin{theorem}{Formula di Taylor con resto di Lagrange p.140}
	SSia
\end{theorem}
\subsubsection{Massimo e Minimo}
\begin{theorem}{Massimo e minimo locale p.143}
	SSia
\end{theorem}
\begin{theorem}{Massimo e minimo globale p.143}
	SSia
\end{theorem}
\begin{theorem}{Teorema di Fermat per funzioni in più variabili p.144}
	SSia
\end{theorem}
\begin{theorem}{Punti critici p.145}
	SSia
\end{theorem}
\begin{theorem}{Punti di sella p.145}
	SSia
\end{theorem}
\begin{theorem}{Test dell'Hessiana p.148}
	SSia
\end{theorem}
\begin{theorem}{Weierstrass p.151}
	SSia
\end{theorem}
\begin{theorem}{Moltiplicatori di Lagrange p.158}
	SSia
\end{theorem}
\break
%%%%%%%%%%%%%%%%%%%%%%%%%%%%%%%%%%%%%%%%%%%%%%%%%%%%%%%%%%%%%%%%%%%%%%%%%%%%%%%%%%%%%%%%%%%%%%%%%%%%%%
\section{Calcolo integrale per funzioni in più variabili}
\begin{theorem}{Funzione integrabile secondo Riemann p.165}
	SSia
\end{theorem}
\begin{theorem}{Funzioni continue p.165}
	SSia
\end{theorem}
\begin{theorem}{Funzioni continue p.165}
	SSia
\end{theorem}
\begin{theorem}{Regione y-semplice p.172}
	SSia
\end{theorem}
\begin{theorem}{Regione x-semplice p.172}
	SSia
\end{theorem}
\begin{theorem}{Formula di Riduzione p.173}
	SSia
\end{theorem}
\begin{theorem}{Proprietà additività domini di integrazione p.173}
	SSia
\end{theorem}

%%%%%%%%%%%%%%%%%%%%%%%%%%%%%%%%%%%%%%%%%%%%%%%%%%%%%%%%%%%%%%%%%%%%%%%%%%%%%%%%%%%%%%%%%%%%%%%%%%%%%%
\end{document}
