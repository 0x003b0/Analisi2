\documentclass{article}  % Tipo di documento
\usepackage[utf8]{inputenc} % Per caratteri accentati
\usepackage[italian]{babel} % Lingua italiana
\usepackage{amsthm}
\usepackage{amssymb}
\usepackage{amsmath}  % simboli matematici avanzati
\usepackage{xcolor} % Per i colori
\usepackage{titlesec} % Per personalizzare i titoli
\usepackage{tikz}
\usetikzlibrary{mindmap,trees}
\usepackage[most]{tcolorbox}
\usepackage{subcaption}  % per avere subfigure
\tcbuselibrary{theorems}
\usepackage{tikz}
\usetikzlibrary{automata, positioning, arrows}
\tcbuselibrary{breakable}
\usepackage{graphicx}
\usepackage[table]{xcolor} % da mettere nel preambolo
\usepackage{mathrsfs} % https://www.ctan.org/pkg/mathrsfs
\graphicspath{ {./media/} }
\usepackage{centernot}
\usepackage[hidelinks]{hyperref}
\usepackage{cancel}


\newtcbtheorem[no counter]{theorem}{Teorema}%
{colback=blue!5, 
colframe=blue!50!black, 
fonttitle=\bfseries,
    breakable,            % permette di spezzare il box su più pagine
    enhanced,
    break at=0pt}{}

\theoremstyle{definition}
\newtheorem{definition}{Definizione}[section]

% Imposto colore delle subsection
\titleformat{\subsection}
  {\normalfont\large\color{red}} % stile del titolo
  {\thesubsection}{1em}{} % numerazione e spaziatura

% Definiamo un nuovo ambiente per gli esempi
\newtcolorbox{esempio}[1][]{
    colback=white,       % colore di sfondo
    colframe=gray,       % colore del bordo
    fonttitle=\bfseries,
    title=#1,
    boxrule=0.5pt,       % spessore del bordo
    arc=4pt,             % angoli arrotondati
    left=4pt, right=4pt, top=4pt, bottom=4pt,
	breakable,            % permette di spezzare il box su più pagine
    enhanced,
    break at=0pt
}

\newtcolorbox{esercizio}[1][]{
    colback=white,       % colore di sfondo
    colframe=green!60!black,       % colore del bordo
    fonttitle=\bfseries,
    title=#1,
    boxrule=0.5pt,       % spessore del bordo
    arc=4pt,             % angoli arrotondati
    left=4pt, right=4pt, top=4pt, bottom=4pt,
    breakable,            % permette di spezzare il box su più pagine
    enhanced,
    break at=0pt
}

\newtcolorbox{osservazioni}[1][]{
    colback=white,       % colore di sfondo
    colframe=yellow!80!orange,       % colore del bordo
    fonttitle=\bfseries,
    title=#1,
    boxrule=0.5pt,       % spessore del bordo
    arc=4pt,             % angoli arrotondati
    left=4pt, right=4pt, top=4pt, bottom=4pt,
    breakable,            % permette di spezzare il box su più pagine
    enhanced,
    break at=0pt
}

% Creo un nuovo ambiente "ragionamento" senza quadratino
\newenvironment{ragionamento}[1][]
  {\begin{proof}[Ragionamento#1]\renewcommand{\qedsymbol}{}\normalfont}
  {\end{proof}}

\title{Analisi 2}
\author{Ede Boanini}
\date{\today}

\begin{document}
\maketitle
\tableofcontents % genera automaticamente l’indice
\newpage
\footnotesize
%%%%%%%%%%%%%%%%%%%%%%%%%%%%%%%%%%%%%%%%%%%%%%%%%%%%%%%%%%%%%%%%%%%%%%%%%%%%%%%%%%%%%%%%%%%%%%%%%%%%%%
\section{Equazioni differenziali}
%%%%%%%%%%%%%%%%%%%%%%%%%%%%%%%%%%%%%%%%%%%%%%%%%%%%%%%%%%%%%%%%%%%%%%%%%%%%%%%%%%%%%%%%%%%%%%%%%%%%%%
\section{Calcolo infintesimale per le curve}
\subsection{Norma di un vettore e proprietà associate}
\begin{theorem}{Formula di Carnot p.55}
    SSia
\end{theorem}
\begin{theorem}{Disuguaglianza di Cauchy-Schwartz p.56}
    SSia
\end{theorem}
\begin{theorem}{Disuguaglianza triangolare p.58}
    SSia
\end{theorem}
\subsubsection{Spazio metrico}
\begin{theorem}{Distanza euclidea p.59}
    SSia
\end{theorem}
\begin{theorem}{Palla aperta p.61}
    SSia
\end{theorem}
\begin{theorem}{Insieme limitato p.64}
    SSia
\end{theorem}
\begin{theorem}{Punto interno p.64}
    SSia
\end{theorem}
\begin{theorem}{Punto esterno p.64}
    SSia
\end{theorem}
\begin{theorem}{Punti di frontiera p.64}
    SSia
\end{theorem}
\begin{theorem}{Punto di accumulazione p.65}
    SSia
\end{theorem}
\begin{theorem}{Insieme aperto p.65}
    SSia
\end{theorem}
\begin{theorem}{Insieme chiuso p.65}
    SSia
\end{theorem}
\begin{theorem}{Continuità p.74}
    SSia
\end{theorem}
\begin{theorem}{Criterio del confronto p.77}
    SSia
\end{theorem}
\break
%%%%%%%%%%%%%%%%%%%%%%%%%%%%%%%%%%%%%%%%%%%%%%%%%%%%%%%%%%%%%%%%%%%%%%%%%%%%%%%%%%%%%%%%%%%%%%%%%%%%%%
\section{Calcolo differenziale per funzioni in più variabili}
\begin{theorem}{Curva regolare p.103}
    SSia
\end{theorem}
\begin{theorem}{Definizione di Derivabilità con il vettore gradiente p.111}
    SSia
\end{theorem}
\begin{theorem}{Derivata Direzionale p.112}
    SSia
\end{theorem}
\subsection{Differenziabilità}
\subsubsection{Derivabilità vs Differenziabilità}
\begin{theorem}{Definizione Differenziabilità p.115}
    SSia
\end{theorem}
\begin{theorem}{Definizione Differenziale p.115}
    SSia
\end{theorem}
\begin{theorem}{Definizione Differenziabilità e Continuità p.121}
    SSia
\end{theorem}
\begin{theorem}{Teorema del differenziale totale p.122}
    SSia
\end{theorem}
\begin{theorem}{Regola della catena p.124}
    SSia
\end{theorem}
\begin{theorem}{Proprietà del differenziale p.125}
    SSia
\end{theorem}
\begin{theorem}{Derivazione della funzione composta p.128}
    SSia
\end{theorem}
\begin{theorem}{Teorema di Schwartz p.132}
    SSia
\end{theorem}
\begin{theorem}{Formula di Taylor con resto di Lagrange p.140}
    SSia
\end{theorem}
\subsubsection{Massimo e Minimo}
\begin{theorem}{Massimo e minimo locale p.143}
    SSia
\end{theorem}
\begin{theorem}{Massimo e minimo globale p.143}
    SSia
\end{theorem}
\begin{theorem}{Teorema di Fermat per funzioni in più variabili p.144}
    SSia
\end{theorem}
\begin{theorem}{Punti critici p.145}
    SSia
\end{theorem}
\begin{theorem}{Punti di sella p.145}
    SSia
\end{theorem}
\begin{theorem}{Test dell'Hessiana p.148}
    SSia
\end{theorem}
\begin{theorem}{Weierstrass p.151}
    SSia
\end{theorem}
\begin{theorem}{Moltiplicatori di Lagrange p.158}
    SSia
\end{theorem}
\break
%%%%%%%%%%%%%%%%%%%%%%%%%%%%%%%%%%%%%%%%%%%%%%%%%%%%%%%%%%%%%%%%%%%%%%%%%%%%%%%%%%%%%%%%%%%%%%%%%%%%%%
\section{Calcolo integrale per funzioni in più variabili}
\begin{theorem}{Funzione integrabile secondo Riemann p.165}
    SSia
\end{theorem}
\begin{theorem}{Funzioni continue p.165}
    SSia
\end{theorem}
\begin{theorem}{Funzioni continue p.165}
    SSia
\end{theorem}
\begin{theorem}{Regione y-semplice p.172}
    SSia
\end{theorem}
\begin{theorem}{Regione x-semplice p.172}
    SSia
\end{theorem}
\begin{theorem}{Formula di Riduzione p.173}
    SSia
\end{theorem}
\begin{theorem}{Proprietà additività domini di integrazione p.173}
    SSia
\end{theorem}

%%%%%%%%%%%%%%%%%%%%%%%%%%%%%%%%%%%%%%%%%%%%%%%%%%%%%%%%%%%%%%%%%%%%%%%%%%%%%%%%%%%%%%%%%%%%%%%%%%%%%%
\end{document}
