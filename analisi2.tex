\documentclass{article}  % Tipo di documento
\usepackage[utf8]{inputenc} % Per caratteri accentati
\usepackage[italian]{babel} % Lingua italiana
\usepackage{amsthm}
\usepackage{amssymb}
\usepackage{amsmath}  % simboli matematici avanzati
\usepackage{xcolor} % Per i colori
\usepackage{titlesec} % Per personalizzare i titoli
\usepackage{tikz}
\usetikzlibrary{mindmap,trees}
\usepackage[most]{tcolorbox}
\usepackage{subcaption}  % per avere subfigure
\tcbuselibrary{theorems}
\usepackage{tikz}
\usetikzlibrary{automata, positioning, arrows}
\tcbuselibrary{breakable}
\usepackage{graphicx}
\usepackage[table]{xcolor} % da mettere nel preambolo
\usepackage{mathrsfs} % https://www.ctan.org/pkg/mathrsfs
\graphicspath{ {./media/} }
\usepackage{centernot}
\usepackage[hidelinks]{hyperref}
\usepackage{cancel}


\newtcbtheorem[no counter]{theorem}{Teorema}%
{colback=blue!5, 
colframe=blue!50!black, 
fonttitle=\bfseries,
    breakable,            % permette di spezzare il box su più pagine
    enhanced,
    break at=0pt}{}

\newtcbtheorem[no counter]{definition}{Definizione}%
{colback=magenta!5, 
colframe=magenta, 
fonttitle=\bfseries,
    breakable,            % permette di spezzare il box su più pagine
    enhanced,
    break at=0pt}{}

% Imposto colore delle subsection
\titleformat{\subsection}
  {\normalfont\large\color{red}} % stile del titolo
  {\thesubsection}{1em}{} % numerazione e spaziatura

% Definiamo un nuovo ambiente per gli esempi
\newtcolorbox{esempio}[1][]{
    colback=white,       % colore di sfondo
    colframe=gray,       % colore del bordo
    fonttitle=\bfseries,
    title=#1,
    boxrule=0.5pt,       % spessore del bordo
    arc=4pt,             % angoli arrotondati
    left=4pt, right=4pt, top=4pt, bottom=4pt,
	breakable,            % permette di spezzare il box su più pagine
    enhanced,
    break at=0pt
}

\newtcolorbox{esercizio}[1][]{
    colback=white,       % colore di sfondo
    colframe=green!60!black,       % colore del bordo
    fonttitle=\bfseries,
    title=#1,
    boxrule=0.5pt,       % spessore del bordo
    arc=4pt,             % angoli arrotondati
    left=4pt, right=4pt, top=4pt, bottom=4pt,
    breakable,            % permette di spezzare il box su più pagine
    enhanced,
    break at=0pt
}

\newtcolorbox{osservazioni}[1][]{
    colback=white,       % colore di sfondo
    colframe=yellow!80!orange,       % colore del bordo
    fonttitle=\bfseries,
    title=#1,
    boxrule=0.5pt,       % spessore del bordo
    arc=4pt,             % angoli arrotondati
    left=4pt, right=4pt, top=4pt, bottom=4pt,
    breakable,            % permette di spezzare il box su più pagine
    enhanced,
    break at=0pt
}

% Creo un nuovo ambiente "ragionamento" senza quadratino
\newenvironment{ragionamento}[1][]
  {\begin{proof}[Ragionamento#1]\renewcommand{\qedsymbol}{}\normalfont}
  {\end{proof}}

\title{Analisi 2}
\author{Ede Boanini}
\date{\today}

\begin{document}
\maketitle
\tableofcontents % genera automaticamente l’indice
\newpage
\footnotesize
%%%%%%%%%%%%%%%%%%%%%%%%%%%%%%%%%%%%%%%%%%%%%%%%%%%%%%%%%%%%%%%%%%%%%%%%%%%%%%%%%%%%%%%%%%%%%%%%%%%%%%
\section{Equazioni differenziali}
\subsection{EDO di I ordine}
\[
	y'=3e^{x}
\]
\subsection{EDO di II ordine}
\[
	2y''+5y=e^x
\]
\subsubsection{EDO di II ordine omogenee}
\subsubsection{EDO di II ordine non omogenee}
\subsubsection{Metodo di somiglianza: $f(x)$ polinomio}
Utile per trovare la soluzione particolare $y_p$ delle EDO di II ordine non omogenee.
\[
	ay''+by'+cy=f(x)
\]
\textcolor{red}{$f(x)$ è un polinomio di grado $n$}:
\begin{esercizio}[Caso 1: $c \neq 0$]
	\textcolor{red}{$c \neq 0$}
	\[
		y''+2y'-8y=x^2+3x+1
	\]
	Qui $f(x)=x^2+3x+1$ che è un polinomio con
	\begin{itemize}
		\item $a=1$
		\item $b=2$
		\item $c=-8$
	\end{itemize}
	$c \neq 0$ quindi devo cercare polinomio di grado $n$ (che corrisponde al grado del polinomio a destra dell'uguale), quindi $n=2$:
	\begin{enumerate}
		\item Scrivo polinomio di grado $n=2$ generico:
		      \[
			      \alpha x^2+\beta x + \gamma
		      \]
		      Questa sarà la forma della soluzione particolare $y_p$
		\item Devo trovare chi sono $\alpha,\beta,\gamma$, come faccio?
		      Calcolo derivata prima e seconda della soluzione particolare: $y_p=\alpha x^2+\beta x + \gamma$ \\
		      \begin{itemize}
			      \item Calcolo derivata prima:
			            \[
				            y'_p=2 \alpha x+\beta
			            \]
			      \item Calcolo derivata seconda:
			            \[
				            y''_p=2 \alpha
			            \]
		      \end{itemize}
		\item Sostituisco $y_p,y'_p, y''_p$ trovati in $y''+2y'-8y$: \\
		      Ho:
		      \[
			      y''+2y'-8y=x^2+3x+1
		      \]
		      Sostituisco $y_p=y,y'_p=y', y''_p=y''$
		      \[
			      (2 \alpha)+2(2 \alpha x+\beta)-8(\alpha x^2+\beta x + \gamma)=x^2+3x+1
		      \]
		      Ora sviluppo i calcoli per trovare $\alpha,\beta,\gamma$:
		      \begin{align*}
			      (2 \alpha)+2(2 \alpha x+\beta)-8(\alpha x^2+\beta x + \gamma)      & = x^2+3x+1 \\
			      2 \alpha+4 \alpha x+2\beta-8 \alpha x^2 -8 \beta x -8 \gamma       & = x^2+3x+1 \\
			      x^2(-8 \alpha)+x(4 \alpha -8 \beta) + 2 \alpha + 2\beta - 8 \gamma & = x^2+3x+1
		      \end{align*}
		      Metto a sistema: \\
		      \textcolor{red}{$x^2(-8 \alpha)$}$+$\textcolor{blue}{$x(4 \alpha -8 \beta)$} $+$ \textcolor{orange}{$2 \alpha + 2\beta - 8 \gamma$} $=$ \textcolor{red}{$x^2$} $+$ \textcolor{blue}{$3x$} $+$ \textcolor{orange}{$1$}
		      \[
			      \begin{cases}
				      -8 \alpha = 1          \\
				      4 \alpha - 8 \beta = 3 \\
				      2 \alpha + 2 \beta - 8 \gamma = 1
			      \end{cases}
		      \]
		      Risolvo il sistema e ottengo:
		      \[
			      \begin{cases}
				      \alpha = - \frac{1}{8} \\
				      \beta = - \frac{7}{16} \\
				      \gamma = - \frac{17}{64}
			      \end{cases}
		      \]
		\item La soluzione particolare $y_p=\alpha x^2+\beta x + \gamma$ dopo aver trovato $\alpha, \beta, \gamma$ sarà:
		      \begin{align*}
			      y_p
			       & = \alpha x^2+\beta x + \gamma                        \\
			       & = - \frac{1}{8} x^2 - \frac{7}{16} x - \frac{17}{64}
		      \end{align*}
	\end{enumerate}
\end{esercizio}

\begin{esercizio}[Caso 2: $c$ uguale a zero]
	\textcolor{red}{$c = 0, \quad b \neq 0$}
	\[
		y''+7y'=x+3
	\]
	$c=0$ quindi devo cercare polinomio di grado $n+1$, poichè il grado di $x+3$ è 1, allora $n+1=1+1=2$, quindi $n=2$:
	\begin{enumerate}
		\item Scrivo polinomio di grado $n=2$ generico:
		      \[
			      y_p=\alpha x^2+\beta x + \gamma
		      \]
		\item Calcolo derivata prima e seconda
		      \begin{itemize}
			      \item $y'_p=2\alpha x+\beta$
			      \item $y''_p=2\alpha$
		      \end{itemize}
		\item Sostituisco nella EDO originale e trovo $\alpha, \beta, \gamma$: \\
		      Sostituisco $y_p,y'_p, y''_p$ trovati in $y''+7y'$:
		      \[
			      y''+7y'=x+3
		      \]
		      \[
			      (2 \alpha)+7(2\alpha x+\beta)=x+3
		      \]
		      \[
			      2 \alpha+14\alpha x+7\beta=x+3
		      \]
		      \[
			      \textcolor{red}{x(14\alpha)}+\textcolor{blue}{2\alpha+7\beta}=\textcolor{red}{x}+\textcolor{blue}{3}
		      \]
		      Metto a sistema:
		      \[
			      \begin{cases}
				      14 \alpha = 1 \\
				      2 \alpha + 7 \beta =3
			      \end{cases}
		      \]
		      Risolvo il sistema e ottengo:
		      \[
			      \begin{cases}
				      \alpha = \frac{1}{14} \\
				      \beta = \frac{20}{49}
			      \end{cases}
		      \]
		\item Ottengo la soluzione particolare: \\
		      La soluzione particolare $y_p=\alpha x^2+\beta x + \gamma$ dopo aver trovato $\alpha, \beta, \gamma$ sarà:
		      \begin{align*}
			      y_p
			       & = \alpha x^2+\beta x + \gamma            \\
			       & = \frac{1}{14} x^2 + \frac{20}{49} x - 0 \\
			       & = \frac{1}{14} x^2 + \frac{20}{49} x
		      \end{align*}
	\end{enumerate}
\end{esercizio}

\begin{esercizio}[Caso 3]
	\textcolor{red}{$c,b = 0$}
	\[
		y''=x^2+4x-7
	\]
	$c,b=0$ quindi devo cercare un polinomio di grado $n+2$, poichè il grado di $x^2+4x-7$ è 2, allora $n+2=2+2=4$. \\ \\
	\textcolor{red}{Ho due metodi}:
	\begin{itemize}
		\item Integrare due volte, oppure
		\item Trovare il polinomio generico di grado $n=4$:
		      \[
			      \alpha x^4+\beta x^3+\gamma x^2+\delta x + \epsilon
		      \]
	\end{itemize}
	\textcolor{blue}{\textbf{Primo metodo: Integrare due volte}} \\
	Qui non trovo la soluzione particolare ma trovo la soluzione generale della EDO:
	\begin{align*}
		\int(x^2+4x-7)dx
		 & = \int x^2dx + \int 4x dx - \int 7 dx \\
		 & = \frac{x^3}{3} + 2x^2 - 7x + c_1
	\end{align*}
	Integro di nuovo:
	\begin{align*}
		\int\left(\frac{x^3}{3} + 2x^2 - 7x + c_1\right)dx
		 & = \int \frac{x^3}{3}dx + \int 2x^2 dx - \int 7x dx + \int c_1 dx \\
		 & = \frac{x^4}{12} + \frac{2x^3}{3} - \frac{7x^2}{2} + c_1x + c_2
	\end{align*}
	Soluzione generale della EDO:
	\[
		\frac{x^4}{12} + \frac{2x^3}{3} - \frac{7x^2}{2} + c_1x + c_2
	\]
	\textcolor{blue}{\textbf{Secondo metodo: metodo di somiglianza}} \\
	Qui trovo solo la soluzione particolare:
	\begin{enumerate}
		\item Scrivo polinomio di grado $n=4$ generico:
		      \[
			      y_p=\alpha x^4+\beta x^3+\gamma x^2+\delta x + \epsilon
		      \]
		\item Calcolo derivata seconda del polinomio generico:
		      \begin{itemize}
			      \item $y'_p=4\alpha x^3+3\beta x^2+2 \gamma x + \delta$
			      \item $y''_p=12\alpha x^2+6\beta x+2 \gamma$
		      \end{itemize}
		\item Sostituisco la derivata seconda trovata nella EDO e trovo $\alpha, \beta, \gamma, \delta, \epsilon$:
		      \[
			      y''=x^2+4x-7
		      \]
		      diventa:
		      \[
			      (12\alpha x^2+6\beta x+2 \gamma)=x^2+4x-7
		      \]
		      \[
			      x^2(12 \alpha)+x(6 \beta)+2 \gamma=x^2+4x-7
		      \]
		      Metto a sistema:
		      \[
			      \begin{cases}
				      12 \alpha = 1 \\
				      6 \beta =4    \\
				      2 \gamma =-7  \\
			      \end{cases}
		      \]
		      Risolvo il sistema e ottengo:
		      \[
			      \begin{cases}
				      \alpha = \frac{1}{12} \\
				      \beta = \frac{2}{3}   \\
				      \gamma = - \frac{7}{2}
			      \end{cases}
		      \]
		\item Ottengo la soluzione particolare:
		      La soluzione particolare $y_p=\alpha x^4+\beta x^3+\gamma x^2+\delta x + \epsilon$ dopo aver trovato $\alpha, \beta, \gamma, \delta, \epsilon$ sarà:
		      \begin{align*}
			      y_p
			       & = \alpha x^4+\beta x^3+\gamma x^2+\delta x + \epsilon           \\
			       & = \frac{1}{12} x^4 + \frac{2}{3} x^3 - \frac{7}{2} x^2 + 0x + 0 \\
			       & = \frac{1}{12} x^4 + \frac{2}{3} x^3 - \frac{7}{2} x^2
		      \end{align*}
	\end{enumerate}
\end{esercizio}

\subsubsection{Metodo di somiglianza: $f(x)$ esponenziale}
Utile per trovare la soluzione particolare $y_p$ delle EDO di II ordine non omogenee.
\[
	ay''+by'+cy=f(x)
\]
\textcolor{red}{$f(x)$ è un esponenziale} \\
\begin{esercizio}[Caso 1: $\alpha$ non è soluzione dell'eq. caratteristica]
	\textcolor{red}{$f(x)=c \cdot e^{\alpha x}$ e $\alpha$ \textcolor{black}{non} è soluzione dell'equazione caratteristica.}
	\[
		y''+7y'+12y=5e^{2x}
	\]
	In questo caso abbiamo che $c=5$ e $\alpha=2$. \\
	\begin{enumerate}
		\item Trovo soluzioni dell'omogenea associata (ponendo $f(x)=0$):
		      \[
			      y''+7y'+12y=0
		      \]
		      Quindi:
		      \[
			      \lambda^2 + 7 \lambda + 12 = 0
		      \]
		      \[
			      \Delta=b^2-4ac=49-4(12)=49-48=1
		      \]
		      \[
			      x_1,x_2=\frac{-b \pm \sqrt{\Delta}}{2a}=\frac{-7 \pm 1}{2}=
		      \]
		      \\
		      \[
			      x_1=-3
		      \]
		      \[
			      x_2=-4
		      \]
		\item $\alpha$ fa parte delle soluzioni dell'equazione caratteristica? \\
		      No, perchè ho ottenuto $x_1=-3,x_2=-4$. Quindi adesso cerco soluzione particolare nella forma:
		      \[
			      y_p=k\cdot e^{\alpha x}
		      \]
		      \begin{enumerate}
			      \item Calcolo derivata prima e seconda di $y_p=k\cdot e^{2x}$:
			            \begin{itemize}
				            \item $y'_p=2ke^{2x}$
				            \item $y''_p=4ke^{2x}$
			            \end{itemize}
			      \item Sostituisco le derivate nella EDO:
			            \[
				            y''+7y'+12y=5e^{2x}
			            \]
			            diventa:
			            \[
				            (4ke^{2x})+7(2ke^{2x})+12(k\cdot e^{2x})=5e^{2x}
			            \]
			            \[
				            ke^{2x}\left(4+ (7 \cdot 2) + 12\right)=5e^{2x}
			            \]
			            \[
				            30ke^{2x}=5e^{2x}
			            \]
			            Ricavo $k$:
			            \[
				            k=\frac{5e^{2x}}{30e^{2x}}=\frac{5}{30}\cdot \frac{e^{2x}}{e^{2x}}=\frac{1}{6} \cdot (e^{2x-2x})= \frac{1}{6} \cdot e^0= \frac{1}{6}
			            \]
			      \item La soluzione particolare $y_p=k\cdot e^{\alpha x}$ diventa:
			            \[
				            y_p=\frac{1}{6}e^{2x}
			            \]
		      \end{enumerate}
	\end{enumerate}
	Quindi la soluzione della EDO sarà $y=y_O+y_p$, ovvero:
	\[
		y=\left(c_1 e^{-3x}+c_2 e^{-4x}\right)+ \frac{1}{6}e^{2x}
	\]
\end{esercizio}

\begin{esercizio}[Caso 2: $\alpha$ è soluzione dell'eq. caratteristica]
	\textcolor{red}{$f(x)=c \cdot e^{\alpha x}$ e $\alpha$ è soluzione dell'equazione caratteristica.}
	\[
		3y''-20y'-7y=4e^{7x}
	\]
	In questo caso abbiamo che $c=4$ e $\alpha=7$. \\
	\begin{enumerate}
		\item Trovo soluzioni dell'omogenea associata (ponendo $f(x)=0$):
		      \[
			      3y''-20y'-7y=0
		      \]
		      Quindi:
		      \[
			      3\lambda^2 - 20 \lambda -7 = 0
		      \]
		      \[
			      \Delta=b^2-4ac=400-4(3)(-7)=400+84=484
		      \]
		      \[
			      x_1,x_2=\frac{-b \pm \sqrt{\Delta}}{2a}=\frac{20 \pm \sqrt{484}}{6}=\frac{-20 \pm 22}{6}=
		      \]
		      \\
		      \[
			      x_1=-\frac{1}{3}
		      \]
		      \[
			      x_2=7
		      \]
		\item $\alpha$ fa parte delle soluzioni dell'equazione caratteristica? \\
		      Sì, perchè ho ottenuto $x_1=-\frac{1}{3},x_2=7$. Quindi adesso cerco soluzione particolare nella forma:
		      \[
			      y_p=kx \cdot e^{\alpha x}
		      \]
		      \begin{enumerate}
			      \item Calcolo derivata prima e seconda di $y_p=kx \cdot e^{7x}$:
			            \begin{itemize}
				            \item \begin{align*}
					                  y'_p
					                   & = \bigl( f' \cdot g + f \cdot g' \bigr) \\[4pt]
					                   & = \bigl( (k x)' \cdot e^{7x}
					                  + k x \cdot (e^{7x})' \bigr)               \\[4pt]
					                   & = k e^{7x} + 7k x e^{7x}
				                  \end{align*}

				            \item \begin{align*}
					                  y''_p
					                   & = \textcolor{red}{(k e^{7x})'} + (7k x e^{7x})'     \\[4pt]
					                   & = \textcolor{red}{7k e^{7x}}
					                  + \bigl( (7k x)' \cdot e^{7x}
					                  + 7k x \cdot (e^{7x})' \bigr)                          \\[4pt]
					                   & = 7k e^{7x} + \left(7k e^{7x} + 49k x e^{7x}\right) \\[4pt]
					                   & = 14k e^{7x}+49k x e^{7x}
				                  \end{align*}
			            \end{itemize}
			      \item Sostituisco le derivate nella EDO e trovo $k$:
			            \[
				            3y''-20y'-7y=4e^{7x}
			            \]
			            diventa:
			            \[
				            3(14k e^{7x}+49k x e^{7x})-20(k e^{7x} + 7k x e^{7x} )-7(kx \cdot e^{7x})=4e^{7x}
			            \]
			            \[
				            42k e^{7x}+147k x e^{7x}-20k e^{7x}-140k x e^{7x}-7kxe^{7x}=4e^{7x}
			            \]
			            \[
				            ke^{7x}(42+147x-20-140x-7x)=4e^{7x}
			            \]
			            \[
				            ke^{7x}(22)=4e^{7x}
			            \]
			            Ricavo $k$:
			            \[
				            22ke^{7x}=4e^{7x}
			            \]
			            \[
				            k=\frac{4e^{7x}}{22e^{7x}}=\frac{4}{22}=\frac{2}{11}
			            \]
			      \item La soluzione particolare $y_p=kx\cdot e^{\alpha x}$ diventa:
			            \[
				            y_p=\frac{2}{11}xe^{7x}
			            \]
		      \end{enumerate}
	\end{enumerate}
	Quindi la soluzione della EDO sarà $y=y_O+y_p$, ovvero:
	\[
		y=\left(c_1 e^{-\frac{1}{3}x}+c_2 e^{7x}\right)+ \frac{2}{11}xe^{7x}
	\]
\end{esercizio}

\begin{esercizio}[Caso 3]
	\textcolor{red}{$f(x)=c \cdot e^{\alpha x}$ e \\ $\alpha= \lambda_1= \lambda_2$ è uguale ad entrambe le soluzioni dell'equazione caratteristica.}
	\[
		y''-12y'+36y=3e^{6x}
	\]
	In questo caso abbiamo che $c=3$ e $\alpha=6$. \\
	\begin{enumerate}
		\item Trovo soluzioni dell'omogenea associata (ponendo $f(x)=0$):
		      \[
			      y''-12y'+36y=0
		      \]
		      Quindi:
		      \[
			      \lambda^2-12\lambda+36=0
		      \]
		      \[
			      \Delta=b^2-4ac=144-144=0
		      \]
		      \[
			      \lambda_1,\lambda_2=\frac{-b \pm 0}{2a}=\frac{12}{2}=6
		      \]
		      Quindi ho due soluzioni coincidenti.
		\item Quando le soluzioni dell'equazione caratteristica coincidono e sono uguali ad $\alpha$, allora cerco soluzione particolare nella forma:
		      \[
			      y_p=kx^2 \cdot e^{\alpha x}
		      \]
		      \begin{enumerate}
			      \item Calcolo derivata prima e seconda di $y_p=kx^2 \cdot e^{6 x}$:
			            \begin{itemize}
				            \item \begin{align*}
					                  y'_p
					                   & = \bigl( f' \cdot g + f \cdot g' \bigr)                \\[4pt]
					                   & = \bigl((kx^2)' \cdot e^{6 x} + kx^2 \cdot (e^{6 x})') \\[4pt]
					                   & = 2kxe^{6 x} + 6kx^2e^{6 x}
				                  \end{align*}
				            \item \begin{align*}
					                  y''_p
					                   & = \bigl( f' \cdot g + f \cdot g' \bigr)                                                                       \\[4pt]
					                   & = \bigl((2kx)' \cdot e^{6 x} + 2kx \cdot (e^{6 x})') + \bigl((6kx^2)' \cdot e^{6 x} + 6kx^2 \cdot (e^{6 x})') \\[4pt]
					                   & = \left(2ke^{6 x} + 12kxe^{6 x}\right)+ \left(12kxe^{6 x} + 36kx^2e^{6 x}\right)                              \\
					                   & = 36kx^2e^{6 x} + 24kxe^{6 x} + 2ke^{6 x}
				                  \end{align*}
			            \end{itemize}
			      \item Sostiuisco le derivate nella EDO iniziale e ricavo $k$:
			            \[
				            y''-12y'+36y=3e^{6x}
			            \]
						diventa:
						\[
						(36kx^2e^{6 x} + 24kxe^{6 x} + 2ke^{6 x})-12(2kxe^{6 x} + 6kx^2e^{6 x})+36(kx^2 e^{6 x})=3e^{6x}
						\]
						\[
						36kx^2e^{6 x} + 24kxe^{6 x} + 2ke^{6 x}-24kxe^{6 x}-72kx^2e^{6 x}+ 36kx^2 e^{6 x}=3e^{6x}
						\]
						\[
						2ke^{6 x}=3e^{6x}
						\]
						\[
						k = \frac{3e^{6x}}{2e^{6x}}=\frac{3}{2}
						\]
			      \item La soluzione particolare $y_p=kx^2e^{\alpha x}$ diventa:
			      \[
				  y_p=\frac{3}{2}x^2e^{6 x}
				  \]
		      \end{enumerate}
	\end{enumerate}
	La soluzione della EDO $y=y_O+y_p$ è:
	\[
	y=\left(c_1e^{6x}+c_2xe^{6x}\right) + \frac{3}{2}x^2e^{6 x}
	\]
\end{esercizio}


\subsubsection{Metodo di somiglianza: $f(x)$ è $\sin$ o $\cos$}
Utile per trovare la soluzione particolare $y_p$ delle EDO di II ordine non omogenee.
\[
	ay''+by'+cy=f(x)
\]
\textcolor{red}{$f(x)$ è $\sin(x)$ o $\cos(x)$} \\
%%%%%%%%%%%%%%%%%%%%%%%%%%%%%%%%%%%%%%%%%%%%%%%%%%%%%%%%%%%%%%%%%%%%%%%%%%%%%%%%%%%%%%%%%%%%%%%%%%%%%%
\section{Calcolo infintesimale per le curve}
\subsection{Calcolo vettoriale}
\begin{definition}{Vettore}
	Il vettore $\vec{x} \in \mathbb{R}^n$ è una n-upla:
	\[
		\vec{x}=\left(x_1,x_2,\cdots,x_n\right)
	\]
\end{definition}
\subsubsection{Norma di un vettore e proprietà associate}
\begin{definition}{Norma di un vettore}
	SSia $\vec{x} \in \mathbb{R}^n$ un vettore. Indico la \textcolor{red}{norma/lunghezza} di un vettore come la radice quadrata del prodotto scalare del vettore $\vec{x}$ per se stesso
	o equivalentemente, la radice quadrata della somma dei suoi componenti al quadrato, il numero reale non negativo:
	\[
		\textcolor{red}{\|\vec{x}\|}=\sqrt{\langle \vec{x}, \vec{x} \rangle}=\sqrt{x_1^2+x_2^2+ \cdots + x_n^2}=\sqrt{\sum_{i=1}^{n}(x_i)^2}
	\]
\end{definition}
\begin{theorem}{Formula di Carnot}
	SSiano $\vec{x},\vec{y} \in \mathbb{R}^n$ due vettori. La norma al quadrato della somma dei due vettori equivale alle loro norme al quadrato più due volte il loro prodotto scalare
	\footnote{Un pò come $(a+b)^2=a^2+b^2+2ab$}:
	\[
		\textcolor{red}{\left\|\vec{x}+\vec{y}\right\|^2}=\left\|\vec{x}\right\|^2+\left\|\vec{y}\right\|^2+2\langle \vec{x}, \vec{y} \rangle
	\]
	\begin{proof}
		Sapendo che $\|\vec{x}\|=\sqrt{\langle \vec{x}, \vec{x} \rangle}$, allora:
		\begin{align*}
			\textcolor{red}{\left\|\vec{x}+\vec{y}\right\|^2}
			 & = \left(\sqrt{\langle \vec{x} + \vec{y}, \vec{x} + \vec{y} \rangle}\right)^2                                                                \\
			 & = {\langle \vec{x} + \vec{y}, \vec{x} + \vec{y} \rangle}                                                                                    \\
			 & \textcolor{cyan}{\text{per bilinearità del prodotto scalare,}}                                                                              \\
			 & \textcolor{cyan}{\text{come se facessi }(a+b)(a+b)=a^2+ab+ba+b^2}                                                                           \\
			 & = \langle \vec{x}, \vec{x} \rangle + \langle \vec{x}, \vec{y} \rangle + \langle \vec{y}, \vec{x} \rangle + \langle \vec{y}, \vec{y} \rangle \\
			 & = \|\vec{x}\|^2+ 2\langle \vec{x}, \vec{y} \rangle + \|\vec{y}\|^2                                                                          \\
			 & = \|\vec{x}\|^2 + \|\vec{y}\|^2 + 2\langle \vec{x}, \vec{y} \rangle
		\end{align*}
	\end{proof}
\end{theorem}
\begin{theorem}{Disuguaglianza di Cauchy-Schwartz p.56}
	SSia
\end{theorem}
\subsubsection{Funzione "lunghezza" di un vettore}
\begin{theorem}{Disuguaglianza triangolare p.58}
	QQuesto teorema usa chauchy Schwartz e formula di carnot per la dimostrazione
\end{theorem}
\subsection{Spazio metrico}
\begin{definition}{Distanza euclidea}
	SSiano $\vec{x},\vec{y} \in \mathbb{R}^n$ due vettori. La \textcolor{red}{distanza euclidea} tra $\vec{x}$ e $\vec{y}$ è il numero reale non negativo definito dalla norma della loro differenza:
	\[
		d(\vec{x},\vec{y})=\left\|\vec{x}-\vec{y}\right\|=\sqrt{\langle \vec{x}-\vec{y}, \vec{x}-\vec{y} \rangle}=\sqrt{\sum_{i=1}^{n}(x_i-y_i)^2}
	\]
\end{definition}
\begin{definition}{Palla aperta}
	S\begin{itemize}
		\item Sia $\vec{x_0} \in \mathbb{R}^n$ fissato
		\item Sia il raggio $r>0$ dove $r \in \mathbb{R}$
	\end{itemize}
	Si definisce \textcolor{red}{palla aperta} o intorno sferico di centro $\vec{x_0}$ e raggio $r$, come l'insieme:
	\[
		B(\vec{x_0},r)=\left\{\vec{x} \in \mathbb{R}^n \mid d(\vec{x}, \vec{x_0})<r\right\} \subseteq \mathbb{R}^n
	\]
\end{definition}
\begin{definition}{Insieme limitato p.64}
	DDefinisco l'insieme $A \subseteq \mathbb{R}^n$ \textcolor{red}{limitato} se $\exists$ una palla aperta di centro $\vec{x_0} \in \mathbb{R}^n$ e raggio $r>0$ in cui $A$ ne risulta interamente contenuto:
	\[
	A \subset B(\vec{x_0},r)
	\]
\end{definition}
\begin{definition}{Punto interno}
	UUn \textcolor{red}{punto interno} di un insieme è un punto per il quale esiste almeno un intorno interamente contenuto nell'insieme. \\
	Sia $x_0 \in A \subseteq \mathbb{R}$ (punto appartenente all'insieme $A$), si dice che $x_0$ è un punto interno ad $A$ se esiste un intorno di centro $x_0$ e raggio $r$
	completamente contenuto in A:
	\[
	\exists r>0 \mid B(x_0,r) \subset A
	\]
\end{definition}
\begin{definition}{Punto esterno}
	UUn \textcolor{red}{punto esterno} ad un insieme è un punto per il quale esiste almeno un intorno interamente contenuto nel complementare dell'insieme. \\
	Sia $x_0 \in \mathbb{R}$, si dice che $x_0$ è un punto esterno ad $A \subseteq \mathbb{R}$ se esiste un intorno di centro $x_0$ e raggio $r$
	completamente contenuto nel complementare di A:
	\[
	\exists r>0 \mid B(x_0,r) \subset \overline{A}
	\]
\end{definition}
\begin{definition}{Punti di frontiera}
	UUn \textcolor{red}{punto di frontiera} (bordo) di un insieme è un punto il cui ogni intorno contiene almeno un punto dell'insieme e del suo complementare. \\
	Sia $x_0 \in \mathbb{R}$, si dice che $x_0$ è un punto di frontiera di $A \subseteq \mathbb{R}$ se ogni intorno di centro $x_0$ e raggio $r$
	contiene almeno un punto di $A$ e allo stesso tempo, almeno un punto di $\overline{A}$:
	\[
	\forall{r}>0, \quad \exists y_1 \in A, \quad \exists y_2 \in \overline{A} \quad \text{t.c.} \quad  y_1,y_2 \in B(x_0,r)
	\]
\end{definition}
\begin{definition}{Punto di accumulazione p.65}
	SSia
\end{definition}
\begin{definition}{Insieme aperto p.65}
	SSia
\end{definition}
\begin{definition}{Insieme chiuso p.65}
	SSia
\end{definition}
\begin{theorem}{Continuità p.74}
	SSia
\end{theorem}
\begin{theorem}{Criterio del confronto p.77}
	SSia
\end{theorem}
\break
%%%%%%%%%%%%%%%%%%%%%%%%%%%%%%%%%%%%%%%%%%%%%%%%%%%%%%%%%%%%%%%%%%%%%%%%%%%%%%%%%%%%%%%%%%%%%%%%%%%%%%
\section{Calcolo differenziale per funzioni in più variabili}
\begin{theorem}{Curva regolare p.103}
	SSia
\end{theorem}
\begin{theorem}{Definizione di Derivabilità con il vettore gradiente p.111}
	SSia
\end{theorem}
\begin{theorem}{Derivata Direzionale p.112}
	SSia
\end{theorem}
\subsection{Differenziabilità}
\subsubsection{Derivabilità vs Differenziabilità}
\begin{theorem}{Definizione Differenziabilità p.115}
	SSia
\end{theorem}
\begin{theorem}{Definizione Differenziale p.115}
	SSia
\end{theorem}
\begin{theorem}{Definizione Differenziabilità e Continuità p.121}
	SSia
\end{theorem}
\begin{theorem}{Teorema del differenziale totale p.122}
	SSia
\end{theorem}
\begin{theorem}{Regola della catena p.124}
	SSia
\end{theorem}
\begin{theorem}{Proprietà del differenziale p.125}
	SSia
\end{theorem}
\begin{theorem}{Derivazione della funzione composta p.128}
	SSia
\end{theorem}
\begin{theorem}{Teorema di Schwartz p.132}
	SSia
\end{theorem}
\begin{theorem}{Formula di Taylor con resto di Lagrange p.140}
	SSia
\end{theorem}
\subsubsection{Massimo e Minimo}
\begin{theorem}{Massimo e minimo locale p.143}
	SSia
\end{theorem}
\begin{theorem}{Massimo e minimo globale p.143}
	SSia
\end{theorem}
\begin{theorem}{Teorema di Fermat per funzioni in più variabili p.144}
	SSia
\end{theorem}
\begin{theorem}{Punti critici p.145}
	SSia
\end{theorem}
\begin{theorem}{Punti di sella p.145}
	SSia
\end{theorem}
\begin{theorem}{Test dell'Hessiana p.148}
	SSia
\end{theorem}
\begin{theorem}{Weierstrass p.151}
	SSia
\end{theorem}
\begin{theorem}{Moltiplicatori di Lagrange p.158}
	SSia
\end{theorem}
\break
%%%%%%%%%%%%%%%%%%%%%%%%%%%%%%%%%%%%%%%%%%%%%%%%%%%%%%%%%%%%%%%%%%%%%%%%%%%%%%%%%%%%%%%%%%%%%%%%%%%%%%
\section{Calcolo integrale per funzioni in più variabili}
\begin{theorem}{Funzione integrabile secondo Riemann p.165}
	SSia
\end{theorem}
\begin{theorem}{Funzioni continue p.165}
	SSia
\end{theorem}
\begin{theorem}{Funzioni continue p.165}
	SSia
\end{theorem}
\begin{theorem}{Regione y-semplice p.172}
	SSia
\end{theorem}
\begin{theorem}{Regione x-semplice p.172}
	SSia
\end{theorem}
\begin{theorem}{Formula di Riduzione p.173}
	SSia
\end{theorem}
\begin{theorem}{Proprietà additività domini di integrazione p.173}
	SSia
\end{theorem}

%%%%%%%%%%%%%%%%%%%%%%%%%%%%%%%%%%%%%%%%%%%%%%%%%%%%%%%%%%%%%%%%%%%%%%%%%%%%%%%%%%%%%%%%%%%%%%%%%%%%%%
\end{document}
